\documentclass[sigconf]{acmart}

% ---------- Required packages (ACM whitelist) ----------
\usepackage{graphicx}
\usepackage{subcaption}
\usepackage{amsmath}
\usepackage{booktabs}
\usepackage{multirow}
\usepackage{hyperref}

% ---------- Metadata ----------
\title{Enhancing [Task/Domain] via Novel Deep Autoencoder Architectures}

\author{Mluleki Mtande}
\affiliation{
  \institution{[Your Institution or Independent Researcher]}
  \city{}
  \country{}
}
\email{[your.email@example.com]}

\begin{abstract}
We present a novel autoencoder-based architecture for [task/domain] that improves latent representation quality through [e.g., residual layers, contrastive objectives, curriculum noise]. Our approach is designed to address common shortcomings in traditional autoencoders such as oversmoothing and information loss in deep bottlenecks. We evaluate our model on [dataset name], demonstrating improved performance in terms of [metrics, e.g., reconstruction loss, classification accuracy] compared to strong baselines. Our work contributes [key contributions] and offers a practical path toward more expressive unsupervised representations.
\end{abstract}

\keywords{Deep Learning, Autoencoder, Representation Learning, [Domain], [Technique]}

\begin{document}

\maketitle

\section{Introduction}
\subsection{Motivation}
[Add background, why this problem matters]
\subsection{Problem Statement}
[What is the core technical challenge?]
\subsection{Contributions}
\begin{itemize}
  \item We propose a novel autoencoder architecture for [task].
  \item We introduce [enhancements, e.g., contrastive loss, curriculum noise].
  \item We conduct thorough empirical evaluation on [dataset].
  \item We release our code at: \url{https://github.com/mlu1/Deep-learning-projects/tree/master/project17}
\end{itemize}
\subsection{Paper Organization}
The paper is organized as follows: Section~\ref{sec:related} reviews related work. Section~\ref{sec:method} describes the proposed model. Experimental setup and results are detailed in Section~\ref{sec:experiments}. We conclude in Section~\ref{sec:conclusion}.

\section{Related Work}
\label{sec:related}
\subsection{Autoencoders in [Domain]}
\subsection{Representation Learning Techniques}
\subsection{Gap Analysis}

\section{Problem Formulation}
\subsection{Mathematical Notation}
\subsection{Task Definition}

\section{Proposed Method}
\label{sec:method}
\subsection{Architecture Overview}
\subsection{Encoder and Decoder Design}
\subsection{Key Enhancements}
\subsection{Training Strategy}

\section{Experimental Setup}
\label{sec:experiments}
\subsection{Datasets and Preprocessing}
\subsection{Baselines}
\subsection{Evaluation Metrics}

\section{Results and Discussion}
\subsection{Quantitative Results}
\subsection{Ablation Study}
\subsection{Qualitative Analysis}
\subsection{Discussion}

\section{Conclusion and Future Work}
\label{sec:conclusion}
\subsection{Summary of Findings}
\subsection{Limitations}
\subsection{Future Directions}

\begin{acks}
[Optional: Funding, collaborators, etc.]
\end{acks}

\bibliographystyle{ACM-Reference-Format}
\bibliography{references}

\end{document}
